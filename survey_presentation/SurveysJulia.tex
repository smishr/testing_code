%%
% Please see https://bitbucket.org/rivanvx/beamer/wiki/Home for obtaining beamer.
%%
\documentclass[aspectratio=169]{beamer}
\usetheme{Boadilla}
\usepackage{hyperref}
\usepackage{verbatim}
%\usepackage[style=apa, citestyle=authoryear, bibstyle=numeric, natbib=true, backend=biber]{biblatex}

%%% Colours and transitions
%%% Taken from https://paulgp.github.io/beamer_tips.pdf
\usepackage[default]{lato}
% These are my colors -- there are many like them, but these ones are mine. 
\definecolor{blue}{RGB}{0,114,178}
\definecolor{red}{RGB}{213,94,0}
\definecolor{yellow}{RGB}{240,228,66}
\definecolor{green}{RGB}{0,158,115}
% I use a beige off white for my background
%\definecolor{MyBackground}{RGB}{255,253,218}
%% Uncomment this if you want to change the background color to something else 
%\setbeamercolor{background canvas}{bg=MyBackground}
%% Change the bg color to adjust your transition slide background color! 
\newenvironment{transitionframe}{\setbeamercolor{background canvas}{bg=white}\begin{frame}}{\end{frame}}

\newenvironment{wideitemize}{\itemize\addtolength{\itemsep}{10pt}}{\enditemize}

\setbeamercolor{frametitle}{fg=blue} 
\setbeamercolor{title}{fg=blue}
\setbeamertemplate{footline}[frame number]
\setbeamertemplate{navigation symbols}{}
\setbeamercolor{itemize item}{fg=blue}
\setbeamercolor{itemize subitem}{fg=blue}
\setbeamercolor{enumerate item}{fg=blue}
\setbeamercolor{enumerate subitem}{fg=blue}
\setbeamercolor{button}{bg=MyBackground,fg=blue}
%%%

\title{Survey Analysis in Julia}
\subtitle{Introduction to Survey.jl}
\author{Shikhar Mishra}
\institute{xKDR Forum}
\date{22 March 2023}

%% START PRES
\begin{document}

\frame{\titlepage}

%\AtBeginSection[]
%  {
%     \begin{frame}<beamer>
%     \frametitle{Table of Contents}
%     \tableofcontents[currentsection]
%     \end{frame}
%  }

\begin{transitionframe}
       \begin{center}
         { \Huge \textcolor{black}{Motivations}}
  \end{center}
\end{transitionframe}

\section{Motivation}

\begin{frame}{What is complex survey analysis?}
\begin{wideitemize}
  \item Surveys are an empirical tool for social, behavioural and experimental sciences
  \item Goal: obtaining estimates for a population by "surveying" a well selected sample
  \item Special techniques available for increasing \underline{precision} and \underline{representation} of the surveyed sample
   	\begin{wideitemize}
  		\item several types of survey "designs" and sampling schemes
  	\end{wideitemize} 
  	
  \item Computing summary statistics from a survey requires  applying mathematical corrections and adjustments
  	\begin{wideitemize}
  		\item eg. population mean is not as simple as arithmetic mean of a numeric vector
  	\end{wideitemize}
  \item A "survey" package exposes an intuitive API to user, and automatically applies formulae and corrections in background
   	\begin{wideitemize}
  		\item In Survey.jl, for population mean of a variable you can do \verb $mean(:variable,design)$
  	\end{wideitemize}
\end{wideitemize}
\end{frame}
%  	\begin{wideitemize}
%  		\item Benchmark for open-source complex survey analysis
%  	\end{wideitemize}	
%  \item R `survey` designed in early 00's for MB's of data 
%	\begin{wideitemize}
%  		\item slow for "large" modern datasets and many class of simulation problems
%  		\item eg. variance estimation using bootstrapping
%  	\end{wideitemize}
%  \item Computation times upto few hours for summary statistics
%  \item Testing against R
%  \item Julia 10k sims, R 1k sims in same time or less.

\begin{frame}{Our engineering journey}
\begin{wideitemize}
  \item Users of R `survey` package
  	\begin{wideitemize}
  		\item Benchmark for open-source complex survey analysis
  	\end{wideitemize}	
  \item R `survey` designed in early 00's for MB's of data 
	\begin{wideitemize}
  		\item slow for "large" modern datasets and many class of simulation problems
  		\item eg. variance estimation using bootstrapping
  		\item computation times upto few hours for summary statistics
  	\end{wideitemize}
\end{wideitemize}
\end{frame}
     
\begin{frame}{Why Julia for complex survey analysis}
	\begin{description}
		\item[Performance]  \textbf{Verbosity} of R/Python meets \textbf{speed}  of a systems language
		\item[Community] Several unmaterialised attempts to create survey analysis package. We received feedback and even contributing PRs on the project.\hyperlink{appendix_end}{\beamergotobutton{1}}
		\item[Dev \& maintenace] Avoid "two-language problem". Survey researchers just want something that works great out of the box.
		\item[Ecosystem] Julia has matured to have substantial statistical computing abilities. Survey.jl is complement to and complemented by the entire data ecosystem.
	\end{description}
%	\begin{verbatim}  and maintenance  + expressivity  interest Statistical e
%			DataFrames, Makie, LinearAlgebra, Optim, Turing, Flux
%	\end{verbatim}
			
\end{frame}


\begin{frame}{Survey.jl}  
  \begin{description}
  \item What our package provides
  \item ...
  \item ... add more here
%  \item[Weights] Sampling weights are the number of individuals in the population that each respondent in the sample is representing. 
  \item Testing against R
  \item Julia 10k sims, R 1k sims in same time or less.
    \end{description}
\end{frame}

\begin{transitionframe}
       \begin{center}
         { \Huge \textcolor{black}{Demo}}
  \end{center}
\end{transitionframe}

\section{Demo}
\begin{frame}
Getting started with Survey.jl - Tutorial part. Try to atleast add the CPHS Survey.jl code into slides. Do small demo if time/audience.
\end{frame}


\begin{transitionframe}
       \begin{center}
         { \Huge \textcolor{black}{Future Roadmap}}
  \end{center}
\end{transitionframe}

\section{Plans}
\begin{frame}
	paste some from the homepage of Documentations. Give overview of future issues and PRs github.
\end{frame}

\appendix

\begin{frame}[label=appendix_end]{Links}
  \begin{wideitemize}
  	\item Julia Discourse posts \href{https://discourse.julialang.org/t/any-package-for-survey-data-analysis/67317}{here} and \href{https://discourse.julialang.org/t/analysis-of-complex-surveys-in-julia/44011}{here}
  	\item Unmaterialised attempts \href{https://github.com/samplics-org/survey.jl}{samplics/survey.jl} and \href{https://github.com/jamanrique/SurveyAnalysis.jl}{jamanrique/SurveyAnalysis.jl}
%    \item[] Now you can make it clear you've done a shitload of work
%      \begin{itemize}
%      \item[]  without having to show everything! \hyperlink{appendix_start}{\beamergotobutton{Back}}
%      \end{itemize}
%    \item[] You label a frame with the \texttt{[label=name]} option, and then point a link to it
%    \item[] You can make an object a link using the \texttt{\textbackslash hyperlink\{label\}\{object\}} command
  \end{wideitemize}
\end{frame}

\end{document}
